\documentclass[FM,BP]{tulthesis}
\usepackage[czech]{babel}
\usepackage[utf8]{inputenc}

\TULtitle{Koncept nízkonákladového sledovacího zařízení pro osobní automobily}{The concept of a low cost tracking device for personal cars}
\TULprogramme{B2646}{Informační technologie}{Information Technology}
\TULbranch{1802R007}{Informační technologie}{Information Technology}
\TULauthor{Tomáš Moravec}
\TULsupervisor{Ing. Lenka Kosková - Třísková}
\TULyear{2016}

\begin{document}
\ThesisStart{male}

\begin{acknowledgement}
Děkuji vedoucí práce paní Ing. Lence Koskové Třískové za odborné vedení a poskytnuté informace při zpracování závěrečné bakalářské práce. Mé poděkování patří též Tomáši Stránskému za sdílení praktických zkušeností s použitými komponenty.
\end{acknowledgement}

\begin{abstractCZ}
Práce se zabývá problematikou sledovacích zařízení pro osobní automobily. V úvodu je definován pojem sledovacího zařízení a požadovaných vlastností. Následuje rešerše aktuální situace na trhu a srovnání užitých konceptů pro realizaci sledovacích zařízení. Na základě rešerše je navržen koncept sledovacího zařízení pro osobní automobily s cílem vytvořit levné a spolehlivé zařízení pro střední a nižší třídu vozů. Výstupem práce je prototyp cílového zařízení, který demonstruje zvolené hardwarové i softwarové řešení.
\end{abstractCZ}

\vspace{2cm}

\begin{abstractEN}
Thesis deals with tracking devices for cars. The introduction defines the concept of surveillance equipment and required attributes. Following a research of the current market situation and comparison of the concepts used for implementing surveillance equipment. Based on the research is designed concept of tracking devices for cars in order to create a cheap and reliable device for middle and lower class cars. The outcome of this work is the prototype of the target device, which demonstrates the chosen hardware and software solution.
\end{abstractEN}

\tableofcontents
\clearpage

\begin{abbrList}
\textbf{GPS} & Global Positioning System, družicový polohovací systém\\
\textbf{GSM} & Groupe Spécial Mobile, globální systém pro mobilní komunikaci\\
\textbf{GPRS} & General Packet Radio Service, služba pro přenos dat v mobilní síti\\
\end{abbrList}

\chapter{Úvod}
Not supported yet.

\section{Sledovací zařízení}
Sledovací zařízení je přístroj, udávající svou vlastní polohu na planetě zemi, pomocí zeměpisné šířky a délky. Poloha se určuje vzhledem k systému družic GPS, které obíhají rovnoměrně rozprostřeny na oběžné dráze země.

\subsection{Historie}
Not supported yet.%GPS, vojenský systém, přestnosti, ...

\subsection{Přítomnost}
Satelitní systém GPS se stal synonymem pro sledovací zařízení a určování polohy. Nicméně dnes není jediným projektem %Glonass, Gallileo

\section{Požadované vlastnosti}
V honu za kvalitním, ale levným sledovacím zařízení, je nutné nalézt kompromis mezi nízkou cenou a pohodlím zákazníka. Abych tohoto docílil, budu se snažit vyzdvihnout vlastnosti, které neovlivní cenu, ale výrazně posílí funkčnost a tím i konkurence schopnost zařízení. Následující vlastnosti pro mne budou hrát důležitou roli při výběru hardwarových kompent, stejně tak mi pomohou se zaměřit na části kódu, které potřebují být optimalizovány nejlépe.

\subsection{Cena}
Koncept bude primárně zaměřen na minimalizaci případné výrobní ceny, s ohledem na zachování všech funkcí, očekávaných od sledovacího zařízení. Aby byl produkt atraktivní, ale levný, je důležité kvalitní zpracování částí, které neovlivní konečnou cenu zařízení, tedy programové části, která je slabým článkem většiny dostupných modelů, v kategorii levných sledovacích zařízení. Velkou úsporu peněz bude hrát připojení na autobaterii, kde levné modely touto možností neoplívají a jsou cenově zatíženy drahými bateriemi, které jsou mnohdy tou nejdražší částí sledovacího zařízení. Nakonec se budu pokoušet odstranit přebytečné funkce, které zbytečně komplikují provoz a přizpívají k vyšší ceně.

\subsection{Spolehlivost}
Zařízení musí být spolehlivé a poskytovat požadované informace za jakýchkoliv podmínek a to i při nízké ceně vybavení. Pokud jde o poruchovost u běžně dostupných modelů, když vynecháme faktory které nelze ovlivnit, například nedostupný signál GPS nebo GSM sítě, je na vině buď ztráta signálu, špatným umístěním ve vozidle (o správném umístění je nutné zákazníka informovat), vybití baterie, softwarová chyba která způsobí zacyklení, nebo při provádění různých operací, ignoruje příchozí SMS. Proto je nutné číst a obsloužit všechny doručené SMS a informovat uživatele o všech skutečnostech, například o výpadku signálu, při žádosti o získání polohy.

\subsection{Bezpečnost}
Žádný z dostupných modelů, v cenové hladině pod 6.000 Kč, nemá možnost nastavení řídících čísel, nebo hesla pro obnovení, při jeho ztrátě. Pokud by případný zloděj, zjistit telefonní číslo, ať už odposlechem poblíž vozidla, nebo z telefonu majitele, je schopný deaktivovat zařízení na dálku a odcizit vozidlo bez sebemenší potuchy majitele. Proto je nutné myslet na bezpečnost jak při pokusu o vypnutí z cizího čísla, tak při případném odpojení od zdroje napájení, které by ve finální verzi jistila integrovaná baterie o velice malé kapacitě, která by zajistila pouze odeslání varovné SMS. Je pravděpodobné že při odpojení od baterie by bylo sledovací zařízení odstraněno z vozidla a delší výdrž by byla zbytečná. Díky tomu se dá ušetřit na ceně, která je z velké části tvořena potřebou velkých baterií.

\subsection{Přívětivost}
SMS s řídícími příkazy musí být jednoduché, uživatelsky přívětivé a snadno zapamatovatelné, aby bylo pro zákazníka ovládání intuitivní a srozumitelné. Nejpřívětivější a nejjednodušší je to, co je pro uživatele nejpřirozenější, tedy běžná slova a věty, stejně tak sledovací zařízení by ve stejné formě mělo i odpovídat. Tedy komunikace mezi uživatelem a sledovacím zařízením by mělo být formou konverzace. Například "Kde jsi", odpověd: "Má poloha bude zaslána během několika minut", později: "Nacházím se na souřadnicích...". Stejně tak instalace zařízení musí být jednoduchá a nesmí jí provázet komplikované nastavování. Vše by mělo fungovat při prvním spuštění automaticky.

\subsection{Vícejazyčnost}
Velkou nevýhodou všech dostupných sledovacích zařízení je orientace pouze jediný jazyk a to Anglický. Jazyka neznalý uživatel může být zmatený a v nestandartních situacích odkázaný pouze na manuál, které některé modely ani nemají. Na vině je primárně to, že většina dostupných modelů je přeprodej čínských zařízení a žádný z českých obchodníků nemá možnost měnit jakékoliv vlastnosti zařízení které prodává. Čínští výrobci u nejlevnějších zařízení mnohdy nedodávají ani manuál, pokud ano, tak čínsky nebo anglicky, takže překlad popřípadě tvorba neexistujícího manuálu závisí na českých prodejcích. Není to problém pouze České republiky, vzhledem k tomu, že se jedná o jednoduchou softwarovou implementaci, základní komunikační rozhraní bude v českém jazyce s možností jednoduchého přepnutí na požadovanou jazykovou lokalizaci. Do finálního produktu bude pouze stačit nahrát dostatečné množství jazykových variant, které vzhledem k nízkému počtu textových řetězců a nenáročnosti jejich uložení v paměti, nebude mít žádný vliv na případnou cenu, které by jinak mohlo být způsobeno nutností navýšení paměti pro data.

\subsection{Nastavitelnost}
Další z nevýhod dostupných produktů je nemožnost jakéhokoliv vlastního nastavení, či personalizace. Ať už se jedná o citlivost senzorů, počet potřebných satelitů pro učení polohy (čím méně satelitů, tím menší přesnost, ale větší šance na získání polohy při špatném signálu), nebo nastavení jednotlivých textových řetězců. Vždy je dobré, aby měl zákazník možnost si své zařízení přizpůsobit dle libosti. Tuto možnost mu zařízení bude nabízet a vzhledem k tomu, že implementace je čistě softwarová a nebude ovlivněna cena.

\subsection{Nenáročnost}
Zařízení musí být nenáročné na údržbu, to znamená i výměna baterií, která by měla být řešena připojením na autobaterii, která dodává energii i když je motor vozidla vypnutý. Na druhou stranu, jeho spotřeba nesmí ovlivnit chod vozu, například vybitím autobaterie. Taková situace může při déle vypnutém motoru nastat velice snadno, protože spotřeba při připojení na sít GPS je pravdu vysoká. Řešením by mělo být připojení pouze do telefonní sítě GSM, kde se bude čekat na příkazy z autorizovaného telefonního čísla. V případě neoprávněného pohybu vozidla, nelze sledovat pohyb vozu pomocí GPS, protože spotřeba by byla příliš vysoká. Je nutné najít alternativní řešení v podobě jednoho z dostupných senzorů. Bežne využívané jsou vibrační senzory, nebo senzor zrychlení, tedy akcelerometr. Ty zajistí připojení do sítě GPS, pouze v případě pohybu.

\chapter{Situace na trhu}
Not supported yet.

\section{Kategorie}
Not supported yet.

\section{Dostupná řešení}
Not supported yet.

\chapter{Koncept}
Not supported yet.

\section{Základní zapojení}
Not supported yet.

\section{Řídící jednotka}
Not supported yet.

\section{Komunikační modul}
Not supported yet.

\section{Senzor pohybu}
Not supported yet.

\section{Softwarový návrh}
Not supported yet.

\chapter{Prototyp}
Not supported yet.

\section{Vývojová deska Arduino}
Not supported yet.

\section{Komunikační modul GPS/GPRS/GSM}
Not supported yet.

\section{Napájení}
Not supported yet.

\chapter{Závěr}
Not supported yet.

\end{document}